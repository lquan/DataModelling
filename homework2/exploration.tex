
The dataset contains $p=26$ attributes, and $n=206$ observations. It also contains some instances with missing attribute values, especially for normalized losses. Because we have a large amount of data ($n>5p$), we chose to simply omit observations with any missing attribute values. If we do this, we still have plenty of data to use---more specifically $n=159$ observations---which is why we chose this method instead of replacing them with for instance variable means or interpolations \cite{missingvalues}.

We have an \emph{exploratory observational} dataset. A tool such as ggobi\footnote{\url{http://www.ggobi.org/}} allows us to visualize the data in different ways to get familiar with the dataset. This way, we interactively searched for extreme outliers and screened explanatory variables.

We found out that the variable `engine-location' only had 3 instances of `rear' (0 after removing instances with missing values) which is why we can easily discard this variable. Furthermore, as we have such a highly dimensional dataset, for our analysis we will mostly only consider the numerical attributes and one categorical variable (that is, `drive-wheels'). We will focus on regression models where the response variable is the price (this
was also the goal in \cite{kibbler} where an instance-based machine learning model was used). Some plots of the response variable are shown in \autoref{fig:priceVsAllNum}. Clearly, this is still quite some data to process and to analyze.

\autoref{fig:corrplot} shows the visualization of the correlation matrix: it is clear that many variables are highly correlated together, for instance, we have an almost perfect correlation between `city.mpg' and `highway.mpg', and `length' and `width'. So this means we have to pay extra attention for multicollinearity effects. %MULTICOLLINEARITY

\autoref{fig:logarithmicTransformation} shows that the price variable should be transformed. We can test this formally using the Shapiro-Wilk normality test on the price.
\begin{verbatim}
Shapiro-Wilk normality test
data:  autos$price
W = 0.829, p-value = 2.351e-12
\end{verbatim}
After a (common) logarithmic transformation,
\begin{verbatim}
Shapiro-Wilk normality test
data:  autos2$price
W = 0.9411, p-value = 3.576e-06
\end{verbatim}
we get a better but still a rather poor result. Using the (more general) BoxCox transformation, where the parameter $\lambda=-0.4$ was found using maximization of the log-likelihood (\autoref{fig:boxcox}), we get a much better result:
\begin{verbatim}
Shapiro-Wilk normality test
data:  autosNum$price
W = 0.9634, p-value = 0.0003285
\end{verbatim}

We also see that the boxplot of drive-wheels versus price (\autoref{fig:drivewheelsbox}) has improved using this transformation.
